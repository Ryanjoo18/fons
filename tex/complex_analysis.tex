\part{Complex Analysis}
\chapter{Complex Numbers}
\section{Definition of $\CC$}
As a set, $\CC=\RR^2=\{(x,y)\mid x,y\in\RR\}$. In other words, elements of $\CC$ are pairs of real numbers.

$\CC$ can be made into a field, by introducing addition and multiplication as
follows:
\begin{enumerate}[label=(\arabic*)]
\item (addition) $(a,b)+(c,d)=(a+c,b+d)$
\item (multiplication) $(a,b)\cdot(c,d)=(ac-bd,ad+bc)$.
\end{enumerate}

$\CC$ is an Abelian group under $+$:
\begin{enumerate}
\item (associativity) $((a, b) + (c, d)) + (e, f) = (a, b) + ((c, d) + (e, f))$.
\item (identity) $(0,0) satisfies (0,0)+(a,b)=(a,b)+(0,0)=(a,b)$.
\item (inverse) Given $(a,b)$, $(-a,-b)$ satisfies $(a,b)+(-a,-b)=(-a,-b)+(a,b)$.
\item (commutativity) $(a,b)+(c,d)=(c,d)+(a,b)$.
\end{enumerate}

$\CC\setminus\{(0,0)\}$ is also an Abelian group under multiplication. It is easy to verify the properties above. Note that $(1,0)$ is the identity and $(a,b)^{-1}=\brac{\frac{a}{a^2+b^2},\frac{-b}{a^2+b^2}}$.

(Distributivity) If $z_1,z_2,z_3\in\CC$, then $z_1(z_2+z_3)=(z_1z_2)+(z_1z_3)$.

Also, we require that $(1,0)\neq(0,0)$, i.e., the additive identity is not the same as the multiplicative identity.

\section{Basic properties of $\CC$}
From now on, we will denote an element of $\CC$ by $z=x+iy$ (the standard notation) instead of $(x, y)$. Hence $(a+ib)+(c+id)=(a+c)+i(b+d)$ and $(a+ib)(c+id) = (ac-bd)+i(ad+bc)$.

$\CC$ has a subfield $\{(x,0)\mid x\in\RR\}$ which is isomorphic to $\RR$. Although the polynomial $x^2+1$ has no zeros over $\RR$, it does over $\CC$: $i^2=-1$.

Alternate descriptions of $\CC$:
\begin{enumerate}
\item $\RR[x]/(x^2+1)$, the quotient of the ring of polynomials with coefficients in $\RR$ by the ideal generated by $x^2+1$.
\item The set of matrices of the form $\begin{pmatrix}a&b\\-b&a\end{pmatrix}$, $a,b\in\RR$, where the operations are standard matrix addition and multiplication.
\end{enumerate}

\begin{exercise}
Prove that the alternate descriptions of $\CC$ are actually isomorphic to $\CC$.
\end{exercise}

\begin{theorem}[Fundamental Theorem of Algebra]
$\CC$ is algebraically closed, i.e., any polynomial $a_nx^n+a_{n-1}x^{n-1}+\cdots+a_1x+a_0$ with coefficients in $\CC$ has a root in $\CC$.
\end{theorem}

This will be proved later, but at any rate the fact that $\CC$ is algebraically closed is one of the most attractive features of working over $\CC$.

\section{$\CC$ as a vector space over $\RR$}
We will now view $\CC$ as a vector space over $\RR$. An $\RR$-vector space is equipped with addition and scalar multiplication so that it is an Abelian group under addition and satisfies:
\begin{enumerate}[label=(\arabic*)]
\item $1z=z$,
\item $a(bz)=(ab)z$,
\item $(a+b)z=az+bz$,
\item $a(z+w)=az+aw$.
\end{enumerate}
Here $a,b\in\RR$ and $z,w\in\CC$. The addition for $\CC$ is as before, and the scalar multiplication is inherited from multiplication, namely $a(x+iy)=(ax)+i(ay)$.

$\CC$ is geometrically represented by identifying it with $\RR^2$. (This is sometimes called the
\textbf{Argand diagram}.)

\section{Complex conjugation and absolute values}
Define \textbf{complex conjugation} as an $\RR$-linear map $\CC\to\CC$ which sends $z=x+iy$ to $z=x-iy$.

Properties of complex conjugation:
\begin{enumerate}[label=(\arabic*)]
\item $\bar{\bar{z}}=z$.
\item $\overline{z+w}=\bar{z}+\bar{w}$.
\item $\overline{z\cdot w}=\bar{z}\cdot\bar{w}$.
\end{enumerate}
Given $z=x+iy\in\CC$, $x$ is called the \textbf{real part} of $C$ and $y$ the \textbf{imaginary part}. We often
denote them by $\Re z$ and $\Im z$:
\[\Re z=\frac{z+\bar{z}}{2},\quad\Im z=\frac{z-\bar{z}}{2i}.\]

Define $|z|=\sqrt{x^2+y^2}$. Observe that, under the identification $z=x+iy\leftrightarrow(x, y)$, $|z|$ is simply the (Euclidean) norm of $(x,y)$.

Properties of absolute values:
\begin{enumerate}[label=(\arabic*)]
\item $|z|^2=z\bar{z}$.
\item $|zw|=|z||w|$.
\item (triangle inequality) $|z+w|\le|z|+|w|$.
\end{enumerate}
The first two are staightforward. The last follows from computing
\[|z+w|^2=(z+w)(\bar{z}+\bar{w})=|z|^2+|w|^2+2\Re z\bar{w}\le|z|^2+|w|^2+2|z\bar{w}|=\brac{|z|+|w|}^2.\]

\chapter{Complex Functions}
\section{Basic Topology}
$\CC$ is a metric space, where
\[d(z,w)=|z-w|.\]
%\begin{definition}
%A sequence $(z_n)_{n\in\NN}\subseteq\CC$ is \vocab{convergent} to $a\in\CC$ if $(|z_n-a|)_{n\in\NN}\subseteq\RR$ is convergent to 0, meaning
%\[\forall\epsilon>0\exists N\in\NN\forall n>N,|z_n-a|<\epsilon.\]
%\end{definition}

\begin{definition}[Open ball]
An \vocab{$\epsilon$-ball} is defined as
\[B_\epsilon(a)\coloneqq\{w\in\CC\mid|w-a|<\epsilon\}.\]
\end{definition}

\begin{definition}[Open set]
$U\subset\CC$ is \vocab{open} if $\forall z\in U\exists\epsilon>0\suchthat B_\epsilon(z)\subset U$.
\end{definition}

The complement of an open set is said to be \vocab{closed}.

\begin{proposition}
Every $\epsilon$-ball is open.
\end{proposition}

\begin{definition}[Limit]
We say that $f(z)$ has \vocab{limit} $A$ as $z\to a$, denoted by $\lim_{z\to a}f(z)=A$ if 
\[\forall\epsilon>0\exists\delta>0\suchthat0<|z-a|<\delta\implies|f(z)-A|<\epsilon.\]
\end{definition}

\begin{definition}[Continuity]
$f:\Omega\to\CC$ for $\Omega\subset\CC$ open is \vocab{continuous} at $z_0$ if $\displaystyle\lim_{z\to z_0}=f(z_0)$.
\end{definition}

\begin{proposition}
$f$ is continuous if and only if $f$ is continuous at all $a\in\Omega$.
\end{proposition}

\begin{proposition}
If $f,g:\Omega\to\CC$ are continuous, then so are $f+g$, $fg$ and $f/g$ (where the last one is defined over $\Omega\setminus\{x\mid g(x)=0\}$).
\end{proposition}

\section{Analytic Functions}
\begin{definition}
$f:\Omega\to\CC$ for $\Omega\subset\CC$ open is \vocab{complex differentiable} at $z_0\in\Omega$ if the complex derivative
\[f^\prime(z_0)\coloneqq\lim_{z\to z_0}\frac{f(z)-f(z_0)}{z-z_0}\]
exists. If $f$ is differentiable at all $z_0\in\Omega$, then $f$ is said to be \vocab{analytic} (holomorphic) on $\Omega$.
\end{definition}

\begin{proposition}
Suppose $f,g:\Omega\to\CC$ are analytic. Then so are $f+g$, $fg$, $f/g$ (where the last one is defined
over $\Omega\setminus\{x\mid g(x)=0\}$).
\end{proposition}

\begin{example}
$f(z)=1$ and $f(z)=z$ are analytic functions from $\CC$ to $\CC$, with derivatives $f^\prime(z)=0$ and $f^\prime(z)=1$ respectively.

Therefore, all polynomials $f(z)=a_nz^n+\cdots+a_1z+a_0$ are analytic, with $f^\prime(z)=na_nz^{n-1}+\cdots+a_1$.
\end{example}

\begin{proposition}
An analytic function is continuous.
\end{proposition}

\begin{proof}
Suppose $f:\Omega\to\CC$ is analytic with derivative 
\[f^\prime(z)=\lim_{h\to0}\frac{f(z+h)-f(z)}{h}.\]
Then
\[\lim_{h\to0}\brac{f(z+h)-f(z)}=f^\prime(z)\lim_{h\to0}h=0.\]
\end{proof}

\section{Cauchy--Riemann Equations}
Write $f(z)=u(z)+iv(z)$, where $u,v:\Omega\to\RR$ are real-valued functions. Suppose $f$ is analytic. We compare two ways of taking the limit $f^\prime(z)$:

First take $h$ to be a real number approaching $0$. Then
\[f^\prime(z)=\pdv{f}{x}=\pdv{u}{x}+i\pdv{v}{x}.\]
Next, take $h$ to be purely imaginary, i.e., let $h=ik$ for some $k\in\RR$. Then
\[f^\prime(z)=\lim_{k\to0}\frac{f(z+ik)-f(z)}{ik}=-i\pdv{f}{y}=-i\pdv{u}{y}+\pdv{v}{y}.\]
Comparing real and imaginary parts, we obtain
\[\pdv{f}{x}=-i\pdv{f}{y},\]
or, equivalently,
\[\pdv{u}{x}=\pdv{v}{y}\quad\text{and}\quad\pdv{v}{x}=-\pdv{u}{y}.\]
The equations above are called the \vocab{Cauchy--Riemann equations}.

Assuming for the time being that $u,v$ have continuous partial derivatives of all orders (and in particular the mixed partials are equal), we can show that
\[\Delta u=\pdv[2]{u}{x}+\pdv[2]{u}{y}=0,\quad\Delta v=\pdv[2]{v}{x}+\pdv[2]{v}{y}=0.\]
Such an equation $\Delta u=0$ is called Laplace's equation and its solution is said to be a harmonic function.
%https://www.math.ucla.edu/~honda/math520/notes.pdf
\subsection{Geometric interpretation}


\subsection{Harmonic functions}

\begin{comment}
$\CC$ is $\RR^2$ with a multiplication. Note that each map $f:\CC\to\CC$ induces a map $f_R:\RR^2\to\RR^2$ (and vice versa).
\begin{example}
Consider $f:\CC\to\CC$, $z\mapsto z^2$.

This is equivalent to $x+iy\mapsto (x+iy)^2=(x^2-y^2)+(2xy)i$.

Thus the mapping is the same as $f_R:\RR^2\to\RR^2$, $(x,y)\mapsto(x^2-y^2,2xy)$.
\end{example}
We want to form a connection between differentiability in $\CC$ and $\RR^2$.
\begin{definition}
A map $f_R:\RR^2\to\RR^2$ is called (totally) differentiable at $\begin{pmatrix}x_0\\ y_0\end{pmatrix}$ if there is a matrix $J\in\RR^{2\times2}$ and a map $\phi:\RR^2\to\RR^2$
\[f_R\brac{\begin{pmatrix}x\\y\end{pmatrix}}=\underbrace{f_R\brac{\begin{pmatrix}x_0\\y_0\end{pmatrix}}+J\brac{\begin{pmatrix}x\\y\end{pmatrix}-\begin{pmatrix}x_0\\ y_0\end{pmatrix}}}_{\text{linear approximation}}+\underbrace{\phi\brac{\begin{pmatrix}x\\ y\end{pmatrix}}}_{\text{error term}}\]
where $\frac{\phi\brac{\begin{pmatrix}x\\ y\end{pmatrix}}}{\norm{\begin{pmatrix}x\\y\end{pmatrix}-\begin{pmatrix}x_0\\ y_0\end{pmatrix}}}\to0$ as $\begin{pmatrix}x\\y\end{pmatrix}\to\begin{pmatrix}x_0\\ y_0\end{pmatrix}$.

$J$ is called the \textbf{Jacobian matrix} of $f_R$ at $\begin{pmatrix}x_0\\y_0\end{pmatrix}\in\RR^2$:
\[J=\begin{pmatrix}
\vdots&\vdots\\
\frac{\partial f_R}{\partial x}&\frac{\partial f_R}{\partial y}\\
\vdots&\vdots
\end{pmatrix}\]
\end{definition}
\begin{example}
Considering the above example, 
\[J=\begin{pmatrix}
2x&-2y\\
2y&2x
\end{pmatrix}.\]
\end{example}
\end{comment}