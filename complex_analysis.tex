\part{Complex Analysis}
%Follow Complex Analysis with Applications Princeton University MAT330 Lecture Notes.

\chapter{Complex Field $\CC$}
The complex plane $\CC$ is a set, which may be identified with the set of points on the plane $\RR^2$. We think of the pair $(x,y)$ together as one complex number $z$ and write it as $z=x+iy$ where $i=\sqrt{-1}$; we call $x$ the real part and $y$ the imaginary part, that is
\[ \Re(x+iy)=x, \quad \Im(x+iy)=y. \]

\begin{definition}[Addition and multiplication]
Addition of complex numbers is done component-wise:
\[ (x_1+iy_1)+(x_2+iy_2)=(x_1+x_2)+i(y_1+y_2). \]
Multiplication of complex numbers follows the rule $i^2=-1$:
\[ (x_1+iy_1)\cdot(x_2+iy_2)=x_1x_2-y_1y_2+i(x_1y_2+y_1x_2). \]
\end{definition}

With these two definitions, we may conclude that arithmetic operations (such as distribution, association, etc) carry over to the complex field using the rule $i^2=-1$. In particular, addition and multiplication are commutative.

\begin{definition}[Division]
For any $z\neq0$, we define $\frac{1}{z}$ as the unique number such that $\frac{1}{z}\cdot z=1$.
\end{definition}

To verify that for any $z\neq0$ such a number exists,
\begin{align*}
\frac{1}{z}&=\frac{1}{x+iy}\\
&=\frac{1}{x+iy}\cdot\frac{x-iy}{x-iy}\\
&=\frac{x-iy}{(x+iy)\cdot(x-iy)}\\
&=\frac{x-iy}{x^2+y^2}\\
&=\frac{x}{x^2+y^2}-i\frac{y}{x^2+y^2}.
\end{align*}

\begin{definition}[Complex conjugate]
An additional algebraic operation on complex numbers is the complex conjugate: $z\mapsto\bar{z}$, which is defined as
\[ \overline{x+iy}=x-iy. \]
\end{definition}

\begin{definition}[Euclidean distance]
Distance in $\CC$ is measured in the same Euclidean way as in $\RR^2$:
\[ |x+iy|=\sqrt{x^2+y^2}. \]
We call $|z|$ the modulus of $z$.
\end{definition}

It is useful to remember these basic relations:
\begin{enumerate}
\item $|\Re(z)|\le|z|$ and $|\Im(z)|\le|z|$.
\item Triangle inequality: $|z+w|\le|z|+|w|$.
\item Reverse triangle inequality: $|z+w|\ge||z|-|w||$.
\item Multiplicativity: $|zw|=|z||w|$.
\end{enumerate}

\begin{definition}[Exponential]
We will make heavy use of the exponential function of a complex number: $\exp:\CC\to\CC$. Recall that $\exp(x)=e^x$ for real $x$, and using the familiar rules of exponents,
\[ \exp(x+iy)=\exp(x)\exp(iy) \]
and we understand the second factor via \emph{Euler's formula}
\[ \exp(i\theta)=\cos\theta+i\sin\theta. \]
Hence
\begin{equation}
\exp(x+iy)=e^x\cos y+ie^x\sin y.
\end{equation}
\end{definition}

Before discussing about polar forms, we define the following function.

\begin{definition}
The function $\mathrm{atan2}:\RR^2\setminus\{0\}\to(-\pi,\pi]$ is defined as $\mathrm{atan2}(y,x)$ being the angle measured between the horizontal axis and a ray from the origin to a point $\begin{bmatrix}x\\y\end{bmatrix}\in\RR^2\setminus\{0\}$. The signs of $x$ and $y$ are used to determine the quadrant of the result and hence possibly shirt $\arctan\brac{\frac{y}{x}}$ by $\pi$ or $-\pi$ as follows:
\[ \mathrm{atan2}(y,x)=\begin{cases}
\arctan\brac{\frac{y}{x}} & x\ge0 \\
\arctan\brac{\frac{y}{x}}+\pi & x<0\land y\ge0\\
\arctan\brac{\frac{y}{x}}-\pi & x<0\land y<0
\end{cases} \]
with the convention $\mathrm{atan2}(y,0)=\pm\frac{\pi}{2}$ for $\frac{y}{|y|}=\pm1$.
\end{definition}

\begin{definition}[Polar form]
The polar form of any complex number is a representation of $x+iy$ using magnitude $\sqrt{x^2+y^2}$ and angle $\arg(z)$
\end{definition}
% https://web.math.princeton.edu/~js129/PDFs/teaching/MAT330_spring_2023/MAT330_Lecture_Notes.pdf



\begin{theorem}[Fundamental Theorem of Algebra]
Let $p(z)=a_0+a_1z+\cdots+a_nz^n$ be a polynomial
of degree $n\ge1$ with real (or complex) coefficients $a_k$. Then the roots of the equation $p(z)=0$ are complex. That is, there are $n$ (not necessarily distinct) complex numbers $\gamma_1,\dots,\gamma_n$ such that
\[ a_0+a_1z+\cdots+a_nz^n=a_n(z-\gamma_1)(z-\gamma_2)\cdots(z-\gamma_n). \]
In particular the theorem shows that a degree $n$ polynomial has, counting repetitions, $n$ roots in $\CC$.
\end{theorem}

The proof of this theorem is far beyond the scope of this text. Note that the theorem only guarantees the existence of the roots of a polynomial somewhere in $\CC$ unlike the quadratic formula which determines exactly the roots. The theorem gives no hints as to where in $\CC$ these roots are to be found.

\section{The Argand Diagram}
The real numbers are often represented on the real line, each point of which corresponds to a unique real number. This number increases as we move from left to right along the line. The complex numbers, having two components, their real and imaginary parts, can be represented on a plane, known as the \vocab{Argand diagram}\footnote{after the Swiss mathematician Jean--Robert Argand (1768 -- 1822).}.

In the Argand diagram the point $(a,b)$ represents the complex number $a+bi$ so that the $x$-axis contains all the real numbers, and so is termed the real axis, and the $y$-axis contains all those complex numbers which are purely imaginary (i.e. have no real part) and so is referred to as the imaginary axis.

Other than Cartesian co-ordinates, a complex number $z$ in the complex plane can also be represented by polar-coordinates $r$ and $\theta$, where $r$ is the distance of $z$ from the origin, and $\theta$ is the angle that the line connecting $z$ to the origin makes with the positive real axis. Then we can write
\begin{equation}
z=x+yi=r\cos\theta+(r\sin\theta)i.
\end{equation}
The relations between Cartesian and polar co-ordinates are simple:
\[ x=r\cos\theta \text{ and } y=r\sin\theta; \quad r=\sqrt{x^2+y^2} \text{ and } \tan\theta=\frac{y}{x}. \]

\begin{definition}
The number $r$ is known as the \vocab{modulus} of $z$, denoted by $|z|$. The number $\theta$ is known as the \vocab{argument} of $z$, denoted by $\arg(z)$. If $z=x+iy$ then
\[ |z|=\sqrt{x^2+y^2}, \quad \sin\arg(z)=\frac{y}{\sqrt{x^2+y^2}}, \quad \cos\arg(z)=\frac{x}{\sqrt{x^2+y^2}}. \]
\end{definition}

Note that the argument of $0$ is undefined. Note also that $\arg(z)$ is defined only up to multiples of $2\pi$. For example, the argument of $1+i$ could be $\frac{\pi}{4}$ or $\frac{9\pi}{4}$ or $-\frac{7\pi}{4}$ etc. Here $\frac{\pi}{4}$ would be the preferred choice as for definiteness we shall take the \vocab{principal values for argument} to be in the range $(-\pi,\pi]$.

\begin{notation}
We shall write $\cs\theta$ for $\cos\theta+i\sin\theta$.
\end{notation}

\begin{exercise}{}{}
Let $z,w \in \CC$. Then prove the following:
\begin{enumerate}[label=(\alph*)]
\item $|zw|=|z||w|$
\item $\absolute{\frac{z}{w}}=\frac{|z|}{|w|}$ if $w\neq0$
\item $z\bar{z}=|z|^2$
\item $|\bar{z}|=|z|$
\item $z\pm w=\bar{z}\pm\bar{w}$
\item $zw=\bar{z}\bar{w}$
\item $\frac{z}{w}=\frac{\bar{z}}{\bar{w}}$ if $w\neq0$
\end{enumerate}
and up to multiples of $2\pi$ then the following equations also hold:
\begin{enumerate}[resume*]
\item $\arg(zw)=\arg(z)+\arg(w)$ if $z,w\neq0$
\item $\arg\brac{\frac{z}{w}}=\arg(z)-\arg(w)$ if $z,w\neq0$
\item $\arg\bar{z}=-\arg(z)$ if $z\neq0$
\end{enumerate}
\end{exercise}

\begin{theorem}[Triangle inequality]
For complex numbers $z$ and $w$,
\[ |z+w|\le|z|+|w| \]
with equality if and only if one of them is $0$ or $\arg(z)=\arg(w)$, that is $z$ and $w$ are on the same ray from the origin.
\end{theorem}

\begin{proof}
Note that for any complex number $z+\bar{z}=2\Re(z)$ and $\Re(z)\le|z|$. So for $z,w \in \CC$,
\[ \frac{z\bar{w}+\bar{z}w}{2}=\Re(z\bar{w})\le|z\bar{w}|=|z||\bar{w}|=|z||w|. \]
Then
\begin{align*}
|z+w|^2 &= (z+w)\overline{(z+w)} \\
&= (z+w)(\bar{z}+\bar{w}) \\
&= z\bar{z}+z\bar{w}+\bar{z}w+w\bar{w} \\
&\le |z|^2+2|z||w|+|w|^2=(|z|+|w|)^2
\end{align*}
to give the required result.
\end{proof}

If the coefficients of polynomial $P(z)$ are real, we have the following result:
\begin{theorem}[Conjugate Root Theorem]
The complex roots of a real polynomial come in pairs. That is, if $z_0$ satisfies the polynomial
equation $a_nz^n+a_{n-1}z^{n-1}+\cdots+a_0=0$, where each $a_i$ is real, then $\bar{z_0}$ is also a root.
\end{theorem}
	
\begin{proof}
Note from the algebraic properties of the conjugate function, proven in the previous proposition, that
\begin{align*}
a_n\overline{z_0}^n+a_{n-1}\overline{z_0}^{n-1}+\cdots+a_1\overline{z_0}+a_0
&= a_n\overline{{z_0}^n}+a_{n-1}\overline{{z_0}^{n-1}}+\cdots+a_1\overline{z_0}+a_0 \\
&= \overline{a_n}\overline{{z_0}^n}+\overline{a_{n-1}}\overline{{z_0}^{n-1}}+\cdots+\overline{a_1}\overline{z_0}+\overline{a_0} \quad \text{[the $a_i$ are real]} \\
&= \overline{a_n{z_0}^n+a_{n-1}{z_0}^{n-1}+\cdots+a_0} \\
&= \overline{0} \quad \text{[as $z_0$ is a root]} \\
&= 0
\end{align*}
\end{proof}

\section{Roots of Unity}
Consider the complex number $z_0=\cs\theta$ where $0\le\theta<2\pi$. The modulus of $z_0$ is $1$, and the argument of $z_0$ is $\theta$.

We have proved above that for $z,w\neq0$ that
\[ |zw|=|z||w| \quad \text{and} \quad \arg(zw)=\arg(z)+\arg(w), \]
up to multiples of $2\pi$. So for any integer $n$, and any $z\neq0$, we have that
\[ |z^n|=|z|^n \quad \text{and} \quad \arg(z^n)=n\arg(z). \]
So the modulus of $(z_0)^n$ is $1$ and the argument of $(z_0)^n$ is $n\theta$, or putting this another way,

\begin{theorem}[De Moivre's Theorem\footnote{Abraham De Moivre (1667--1754), a French protestant who fled religious persecution in France to move to England, is best remembered for this formula but he also made important contributions in probability which appeared in his The Doctrine Of Chances (1718).}]
For a real number $\theta$ and integer $n$ we have that
\[ \cos n\theta+i\sin n\theta=(\cos\theta+i\sin\theta)^n. \]
\end{theorem}

\begin{proof}
The proof easily follows from mathematical induction.
\end{proof}

Alternatively, this is a simple consequence of Euler's formula:
\[ (\cos\theta + i\sin\theta)^n = (e^{i\theta})^n = e^{ni\theta} = \cos n\theta + i\sin n\theta \]
However De Moivre proved his theorem before Euler came up with Euler's formula.

For rational exponent instead of integer exponent,
\[ (\cos\theta+i\sin\theta)^\frac{p}{q}=\cos\frac{p\theta+2k\pi}{q}+i\sin\frac{p\theta+2k\pi}{q} \]
where $p,q\in\ZZ$, $q>0$, $k=0,1,2,\dots,q-1$.

Applications ($z=\cos\theta+i\sin\theta$)
\begin{itemize}
\item $z^n+z^{-n}=2\cos n\theta$
\item $z^n-z^{-n}=2i\sin n\theta$
\item $z+\frac{1}{z}=2\cos\theta$
\item $z-\frac{1}{z}=2i\sin\theta$
\end{itemize}
Uses - to find multiple powers of trigonometry in terms of multiple angles of trigo, and vice versa via binomial expansion

\begin{exercise}{}{}
Find the exact value of $(1+\sqrt{3}i)^6$.
\end{exercise}

\begin{proof}[Solution]
	\begin{align*}
		(1+\sqrt{3}i)^6 &= \sqbrac{2\brac{\cos\frac{\pi}{3}+i\sin\frac{\pi}{3}}}^6 \\
		&= 2^6 (\cos2\pi+i\sin2\pi) \quad \text{[by de Moivre]} \\
		&= \boxed{64}
	\end{align*}
\end{proof}

\begin{exercise}{}{}
Use De Moivre's Theorem to show that
\begin{enumerate}[label=(\alph*)]
\item $\cos5\theta=16\cos^5\theta-20\cos^3\theta+5\cos\theta$, and that
\item $\sin5\theta=(16\cos^4\theta-12\cos^2\theta+1)\sin\theta$
\end{enumerate}
\end{exercise}

\begin{theorem}[Euler's Formula]
For a complex number $z$ with modulus $r$ and argument $\theta$, $z$ can be expressed as
\[ z=re^{i\theta} = r(\cos\theta+i\sin\theta). \]
\end{theorem}

\begin{proof}
	Recall that the Taylor Series for $e^x$ is given by
	\[ e^x = \sum_{n=0}^\infty \frac{x^n}{n!} = 1+x+\frac{x^2}{2!}+\frac{x^3}{3!}+\frac{x^4}{4!}+\frac{x^5}{5!}+\cdots \]
	Substituting $ix$ for $x$ gives us
	\[ e^{ix} = 1+ix-\frac{x^2}{2!}-\frac{ix^3}{3!}+\frac{x^4}{4!}+\frac{ix^5}{5!}+\cdots \]
	Separating real and imaginary parts,
	\[ e^{ix} = \brac{1-\frac{x^2}{2!}+\frac{x^4}{4!}+\cdots} + i\brac{x-\frac{x^3}{3!}+\frac{x^5}{5!}+\cdots} \]
	Observe that the real part is the Taylor series of $\cos x$, while the imaginary part is the Taylor series of $\sin x$. Hence proven.
\end{proof}

Applications
\begin{itemize}
\item $e^{in\theta}+e^{-in\theta}=2\cos n\theta$
\item $e^{in\theta}-e^{-in\theta}=2i\sin n\theta$
\item $e^{i\alpha}+e^{i\beta}=e^{i\frac{\alpha+\beta}{2}}\brac{e^{i\frac{\alpha-\beta}{2}}+e^{-i\frac{\alpha-\beta}{2}}}=e^{i\frac{\alpha+\beta}{2}}2i\cos\frac{\alpha-\beta}{2}$
\item $e^{i\alpha}-e^{i\beta}=e^{i\frac{\alpha+\beta}{2}}\brac{e^{i\frac{\alpha-\beta}{2}}-e^{-i\frac{\alpha-\beta}{2}}}=e^{i\frac{\alpha+\beta}{2}}2i\sin\frac{\alpha-\beta}{2}$
\end{itemize}
Used to prove summation of trigonometry (sigma notation)

\paragraph{Inverse Euler formulae}
Euler's formula gives a complex exponential in terms of sines and cosines. We can turn this around to get the \vocab{inverse Euler formulae}. Euler’s formula says:
\[ e^{it}=\cos t+i\sin t \quad \text{and} \quad e^{-it}=\cos t=i\sin t. \]
By adding and subtracting we get
\[ \cos t=\frac{e^{it}+e^{-it}}{2} \quad \text{and} \quad \sin t=\frac{e^{it}-e^{-it}}{2i}. \]

\paragraph{Complex replacement}
In the next example we will illustrate the technique of complexification or complex replacement. This can be used to simplify a trigonometric integral. It will come in handy when we need to compute certain integrals.

\begin{exercise}{}{}
Use complex replacement to compute
\[ I = \int e^x \cos 2x \dd{x} \]
\end{exercise}
\begin{solution}
From Euler's formula we have
\[ e^{2ix} = cos(2x) + isin(2x) \]
so $\cos 2x = \Re(e^{2ix})$. The complex replacement trick is to replace $\cos 2x$ by $e^{2ix}$. We get
\[ I_c = \int (e^x \cos 2x + ie^x\sin 2x) \dd{x}, \quad I = \Re(I_c). \]
Computing $I_c$ is straightforward:
\[ I_c = \int e^xe^{2ix} \dd{x} = \int e^{x(1+2i)} \dd{x} = \frac{e^{x(1+2i)}}{1+2i}. \]
Here we will do the computation first in rectangular coordinates
\begin{align*}
I_c &= \frac{e^{x(1+2i)}}{1+2i} \cdot \frac{1-2i}{1-2i} \\
&= \frac{e^x(\cos 2x+i\sin 2x)(1-2i)}{5} \\
&= \frac{1}{5}e^x[(\cos2x+2\sin2x) + i(-2\cos2x+\sin2x)]
\end{align*}
So 
\[ I = \Re(I_c) = \boxed{\frac{1}{5}e^x(\cos2x+2\sin2x)} \]
\end{solution}

\paragraph{$n$-th roots of unity}
We are going to need to be able to find the $n$-th \vocab{roots of unity}, i.e. solve equations of the form
\[ z^n=c \]
where $c$ is a given complex number. This can be done most conveniently by expressing $c$
and $z$ in polar form:
\[ c=Re^{i\phi}, \quad z=re^{i\theta}.\]
Then, upon substituting, we have to solve
\[ r^n e^{in\theta} = Re^{i\phi} \]
For the complex numbers on the left and right to be equal, their absolute values must be same and the arguments can only differ by an integer multiple of $2\pi$, which gives
\[ r=R^\frac{1}{n} \quad \text{and} \quad n\theta=\phi+2k\pi, k=0,\pm1,\pm2,\dots \]
Solving for $\theta$, we have
\begin{equation}
\theta = \frac{1}{n}(\phi+2k\pi)
\end{equation}
In general $c=Re^{i\phi}$ has $n$ distinct $n$-th roots:
\[ z_k = r^{1/n}e^{i\phi/n+i 2\pi(k/n)} \]
for $k=0,1,2,\dots,n-1$.


nth roots of unity
\[ z^n=1 \implies z=e^{i\frac{2k\pi}{n}}=\cos\frac{2k\pi}{n}+i\sin\frac{2k\pi}{n} \]
where $k=0,1,2,\dots,n-1$ or $k=0,\pm1,\pm2,\dots$ (n integers)

nth roots of equation
\[ z^n=(a+bi)^p=(re^{i\theta})^p=r^pe^{ir\theta}=r^pe^{i(r\theta+2k\pi)} \implies z=r^\frac{p}{n}e^{i\frac{p+2k\pi}{q}} \]
where $k=0,1,2,\dots,n-1$ or $k=0,\pm1,\pm2,\dots$ (n integers)

\paragraph{Geometry of $n$-th roots}
Roots are always spaced evenly around a circle centred at the origin. For example, the 5th roots of $1+i$ are spaced at increments of $\frac{2\pi}{5}$ radians around the circle of radius $2^\frac{1}{5}$.

Note also that the roots of real numbers always come in conjugate pairs.

%\subsection{Loci in the Complex Plane} Loci of complex numbers, inequalities in loci of complex numbers
%In the Argand diagram, $z=a+bi$ is represented by the point $(a,b)$. If the values of $a$ and $b$ vary according to some given condition, the set of all possible points in the Argand diagram will describe some line or curve, known as the \vocab{locus} of $z$.

\section{Their Geometry}

A locus is a set of points that satisfy given conditions (take $z=x+iy$)

The equation
\[ |z-z_1|=r \]
represents a circle with centre at $(x_1,y_1)$ and radius $r$. The Cartesian equation of the locus is $(x-x_1)^2+(y-y_1)^2=r^2$.

The equation 
\[ |z-z_1|=|z-z_2| \]
represents the perpendicular bisector of the line joining the points $(x_1,y_1)$ and $(x_2,y_2)$. The Cartesian equation will be a linear equation.

The equation
\[ \arg(z-z_1)=\theta \]
represents a half-line from $A(x_1,y_1)$ (excluding $A$) making an angle $\theta$ with the positive real axis.

Inequalities in loci of complex numbers
$|z-z_1|\le r$ points inside and on the circle
$|z-z_1|\le|z-z_2|$ points nearer to $z_1$ and points on perp bisector
$\alpha\le\arg(z-z_1)\le\beta$ points between and on the half-lines
sketching - draw lines / circles dotted if equality does not hold

For other loci - assume $z=x+iy$ and then deduce its nature



% After finishing notes on complex numbers: https://courses.maths.ox.ac.uk/pluginfile.php/36995/mod_resource/content/2/complex.pdf